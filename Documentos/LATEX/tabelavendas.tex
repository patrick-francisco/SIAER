%%  This section checks if we are begin input into another file or  %%
%%  the file will be compiled alone. First use a macro taken from   %%
%%  the TeXbook ex 7.7 (suggestion of Han-Wen Nienhuys).            %%
\def\ifundefined#1{\expandafter\ifx\csname#1\endcsname\relax}


%%  Check for the \def token for inputed files. If it is not        %%
%%  defined, the file will be processed as a standalone and the     %%
%%  preamble will be used.                                          %%
\ifundefined{inputGnumericTable}

%%  We must be able to close or not the document at the end.        %%
	\def\gnumericTableEnd{\end{document}}


%%%%%%%%%%%%%%%%%%%%%%%%%%%%%%%%%%%%%%%%%%%%%%%%%%%%%%%%%%%%%%%%%%%%%%
%%                                                                  %%
%%  This is the PREAMBLE. Change these values to get the right      %%
%%  paper size and other niceties. Uncomment the landscape option   %%
%%  to the documentclass defintion for standalone documents.        %%
%%                                                                  %%
%%%%%%%%%%%%%%%%%%%%%%%%%%%%%%%%%%%%%%%%%%%%%%%%%%%%%%%%%%%%%%%%%%%%%%

	\documentclass[12pt%
			  %,landscape%
                    ]{report}
       \usepackage[latin1]{inputenc}
	\usepackage{fullpage}
	\usepackage{color}
       \usepackage{array}
	\usepackage{longtable}
       \usepackage{calc}
       \usepackage{multirow}
       \usepackage{hhline}
       \usepackage{ifthen}

	\begin{document}


%%  End of the preamble for the standalone. The next section is for %%
%%  documents which are included into other LaTeX2e files.          %%
\else

%%  We are not a stand alone document. For a regular table, we will %%
%%  have no preamble and only define the closing to mean nothing.   %%
    \def\gnumericTableEnd{}

%%  If we want landscape mode in an embedded document, comment out  %%
%%  the line above and uncomment the two below. The table will      %%
%%  begin on a new page and run in landscape mode.                  %%
%       \def\gnumericTableEnd{\end{landscape}}
%       \begin{landscape}


%%  End of the else clause for this file being \input.              %%
\fi

%%%%%%%%%%%%%%%%%%%%%%%%%%%%%%%%%%%%%%%%%%%%%%%%%%%%%%%%%%%%%%%%%%%%%%
%%                                                                  %%
%%  The rest is the gnumeric table, except for the closing          %%
%%  statement. Changes below will alter the table's appearance.     %%
%%                                                                  %%
%%%%%%%%%%%%%%%%%%%%%%%%%%%%%%%%%%%%%%%%%%%%%%%%%%%%%%%%%%%%%%%%%%%%%%

\providecommand{\gnumericmathit}[1]{#1} 
%%  Uncomment the next line if you would like your numbers to be in %%
%%  italics if they are italizised in the gnumeric table.           %%
%\renewcommand{\gnumericmathit}[1]{\mathit{#1}}
\providecommand{\gnumericPB}[1]%
{\let\gnumericTemp=\\#1\let\\=\gnumericTemp\hspace{0pt}}
 \ifundefined{gnumericTableWidthDefined}
        \newlength{\gnumericTableWidth}
        \newlength{\gnumericTableWidthComplete}
        \newlength{\gnumericMultiRowLength}
        \global\def\gnumericTableWidthDefined{}
 \fi
%% The following setting protects this code from babel shorthands.  %%
 \ifthenelse{\isundefined{\languageshorthands}}{}{\languageshorthands{english}}
%%  The default table format retains the relative column widths of  %%
%%  gnumeric. They can easily be changed to c, r or l. In that case %%
%%  you may want to comment out the next line and uncomment the one %%
%%  thereafter                                                      %%
\providecommand\gnumbox{\makebox[0pt]}
%%\providecommand\gnumbox[1][]{\makebox}

%% to adjust positions in multirow situations                       %%
\setlength{\bigstrutjot}{\jot}
\setlength{\extrarowheight}{\doublerulesep}

%%  The \setlongtables command keeps column widths the same across  %%
%%  pages. Simply comment out next line for varying column widths.  %%
\setlongtables

\setlength\gnumericTableWidth{%
	74pt+%
	165pt+%
	71pt+%
	80pt+%
	90pt+%
0pt}
\def\gumericNumCols{5}
\setlength\gnumericTableWidthComplete{\gnumericTableWidth+%
         \tabcolsep*\gumericNumCols*2+\arrayrulewidth*\gumericNumCols}
\ifthenelse{\lengthtest{\gnumericTableWidthComplete > \linewidth}}%
         {\def\gnumericScale{\ratio{\linewidth-%
                        \tabcolsep*\gumericNumCols*2-%
                        \arrayrulewidth*\gumericNumCols}%
{\gnumericTableWidth}}}%
{\def\gnumericScale{1}}

%%%%%%%%%%%%%%%%%%%%%%%%%%%%%%%%%%%%%%%%%%%%%%%%%%%%%%%%%%%%%%%%%%%%%%
%%                                                                  %%
%% The following are the widths of the various columns. We are      %%
%% defining them here because then they are easier to change.       %%
%% Depending on the cell formats we may use them more than once.    %%
%%                                                                  %%
%%%%%%%%%%%%%%%%%%%%%%%%%%%%%%%%%%%%%%%%%%%%%%%%%%%%%%%%%%%%%%%%%%%%%%

\ifthenelse{\isundefined{\gnumericColB}}{\newlength{\gnumericColB}}{}\settowidth{\gnumericColB}{\begin{tabular}{@{}p{74pt*\gnumericScale}@{}}x\end{tabular}}
\ifthenelse{\isundefined{\gnumericColC}}{\newlength{\gnumericColC}}{}\settowidth{\gnumericColC}{\begin{tabular}{@{}p{165pt*\gnumericScale}@{}}x\end{tabular}}
\ifthenelse{\isundefined{\gnumericColD}}{\newlength{\gnumericColD}}{}\settowidth{\gnumericColD}{\begin{tabular}{@{}p{71pt*\gnumericScale}@{}}x\end{tabular}}
\ifthenelse{\isundefined{\gnumericColE}}{\newlength{\gnumericColE}}{}\settowidth{\gnumericColE}{\begin{tabular}{@{}p{80pt*\gnumericScale}@{}}x\end{tabular}}
\ifthenelse{\isundefined{\gnumericColF}}{\newlength{\gnumericColF}}{}\settowidth{\gnumericColF}{\begin{tabular}{@{}p{90pt*\gnumericScale}@{}}x\end{tabular}}

\begin{longtable}[c]{%
	b{\gnumericColB}%
	b{\gnumericColC}%
	b{\gnumericColD}%
	b{\gnumericColE}%
	b{\gnumericColF}%
	}
\caption[Plano de Vendas para 5 anos]{Plano de Vendas para 5 anos}\label{tablePlanodeVendas} \\

%%%%%%%%%%%%%%%%%%%%%%%%%%%%%%%%%%%%%%%%%%%%%%%%%%%%%%%%%%%%%%%%%%%%%%
%%  The longtable options. (Caption, headers... see Goosens, p.124) %%
%	\caption{The Table Caption.}             \\	%
% \hline	% Across the top of the table.
%%  The rest of these options are table rows which are placed on    %%
%%  the first, last or every page. Use \multicolumn if you want.    %%

%%  Header for the first page.                                      %%
%	\multicolumn{5}{c}{The First Header} \\ \hline 
%	\multicolumn{1}{c}{colTag}	%Column 1
%	&\multicolumn{1}{c}{colTag}	%Column 2
%	&\multicolumn{1}{c}{colTag}	%Column 3
%	&\multicolumn{1}{c}{colTag}	%Column 4
%	&\multicolumn{1}{c}{colTag}	\\ \hline %Last column
%	\endfirsthead

%%  The running header definition.                                  %%
%	\hline
%	\multicolumn{5}{l}{\ldots\small\slshape continued} \\ \hline
%	\multicolumn{1}{c}{colTag}	%Column 1
%	&\multicolumn{1}{c}{colTag}	%Column 2
%	&\multicolumn{1}{c}{colTag}	%Column 3
%	&\multicolumn{1}{c}{colTag}	%Column 4
%	&\multicolumn{1}{c}{colTag}	\\ \hline %Last column
%	\endhead

%%  The running footer definition.                                  %%
%	\hline
%	\multicolumn{5}{r}{\small\slshape continued\ldots} \\
%	\endfoot

%%  The ending footer definition.                                   %%
%	\multicolumn{5}{c}{That's all folks} \\ \hline 
%	\endlastfoot
%%%%%%%%%%%%%%%%%%%%%%%%%%%%%%%%%%%%%%%%%%%%%%%%%%%%%%%%%%%%%%%%%%%%%%

\hhline{-----}
	 \gnumericPB{\raggedright}\textbf{Per�odo}
	&\gnumericPB{\raggedright}\textbf{Produto}
	&\gnumericPB{\centering}\textbf{Volume (unidades)}
	&\gnumericPB{\raggedleft}\textbf{Pre�o (R\$)}
	&\gnumericPB{\raggedleft}\textbf{Receita ( R\$)}
\\
\hhline{-----}
	 \gnumericPB{\centering}\gnumbox{2011}
	&\gnumericPB{\raggedright}\gnumbox[l]{Licen�a software para guich�}
	&\gnumericPB{\raggedleft}\gnumbox[r]{12}
	&\gnumericPB{\raggedleft}\gnumbox[r]{$              150 $}
	&\gnumericPB{\raggedleft}\gnumbox[r]{$              1800 $}
\\
	 \gnumericPB{\centering}\gnumbox{2011}
	&\gnumericPB{\raggedright}\gnumbox[l]{Licen�a software para �nibus}
	&\gnumericPB{\raggedleft}\gnumbox[r]{30}
	&\gnumericPB{\raggedleft}\gnumbox[r]{$              50 $}
		&\gnumericPB{\raggedleft}\gnumbox[r]{$              1500 $}
\\
 \gnumericPB{\centering}\gnumbox{\textbf{2011}}
	&\gnumericPB{\raggedright}\gnumbox[l]{\textbf{Sub total - 2011}}
	&\gnumericPB{\raggedleft}\gnumbox[r]{}
	&\gnumericPB{\raggedleft}\gnumbox[r]{}
	&\gnumericPB{\raggedleft}\gnumbox[r]{\textbf{3300}}
\\	
	 \gnumericPB{\centering}\gnumbox{2012}
	&\gnumericPB{\raggedright}\gnumbox[l]{Licen�a software para guich�}
	&\gnumericPB{\raggedleft}\gnumbox[r]{120}
	&\gnumericPB{\raggedleft}\gnumbox[r]{$              160 $}
	&\gnumericPB{\raggedleft}\gnumbox[r]{$              19200 $}
\\
	 \gnumericPB{\centering}\gnumbox{2012}
	&\gnumericPB{\raggedright}\gnumbox[l]{Licen�a software para �nibus}
	&\gnumericPB{\raggedleft}\gnumbox[r]{360}
	&\gnumericPB{\raggedleft}\gnumbox[r]{$              50 $}
	&\gnumericPB{\raggedleft}\gnumbox[r]{$              18000 $}
\\
 \gnumericPB{\centering}\gnumbox{\textbf{2012}}
	&\gnumericPB{\raggedright}\gnumbox[l]{\textbf{Sub total - 2012}}
	&\gnumericPB{\raggedleft}\gnumbox[r]{}
	&\gnumericPB{\raggedleft}\gnumbox[r]{}
	&\gnumericPB{\raggedleft}\gnumbox[r]{\textbf{37200}}
\\	
	 \gnumericPB{\centering}\gnumbox{2013}
	&\gnumericPB{\raggedright}\gnumbox[l]{Licen�a software para guich�}
	&\gnumericPB{\raggedleft}\gnumbox[r]{240}
	&\gnumericPB{\raggedleft}\gnumbox[r]{$              170 $}
	&\gnumericPB{\raggedleft}\gnumbox[r]{$              40800 $}
\\
	 \gnumericPB{\centering}\gnumbox{2013}
	&\gnumericPB{\raggedright}\gnumbox[l]{Licen�a software para �nibus}
	&\gnumericPB{\raggedleft}\gnumbox[r]{1200}
	&\gnumericPB{\raggedleft}\gnumbox[r]{$              55 $}
		&\gnumericPB{\raggedleft}\gnumbox[r]{$              66000 $}
\\
 \gnumericPB{\centering}\gnumbox{\textbf{2013}}
	&\gnumericPB{\raggedright}\gnumbox[l]{\textbf{Sub total - 2013}}
	&\gnumericPB{\raggedleft}\gnumbox[r]{}
	&\gnumericPB{\raggedleft}\gnumbox[r]{}
	&\gnumericPB{\raggedleft}\gnumbox[r]{\textbf{106800}}
\\
	 \gnumericPB{\centering}\gnumbox{2014}
	&\gnumericPB{\raggedright}\gnumbox[l]{Licen�a software para guich�}
	&\gnumericPB{\raggedleft}\gnumbox[r]{600}
	&\gnumericPB{\raggedleft}\gnumbox[r]{$              200 $}
	&\gnumericPB{\raggedleft}\gnumbox[r]{$              120000 $}
\\
	 \gnumericPB{\centering}\gnumbox{2014}
	&\gnumericPB{\raggedright}\gnumbox[l]{Licen�a software para �nibus}
	&\gnumericPB{\raggedleft}\gnumbox[r]{3600}
	&\gnumericPB{\raggedleft}\gnumbox[r]{$              60 $}
	&\gnumericPB{\raggedleft}\gnumbox[r]{$              216000 $}
\\
 \gnumericPB{\centering}\gnumbox{\textbf{2014}}
	&\gnumericPB{\raggedright}\gnumbox[l]{\textbf{Sub total - 2014}}
	&\gnumericPB{\raggedleft}\gnumbox[r]{}
	&\gnumericPB{\raggedleft}\gnumbox[r]{}
	&\gnumericPB{\raggedleft}\gnumbox[r]{\textbf{336000}}
\\
	 \gnumericPB{\centering}\gnumbox{2015}
	&\gnumericPB{\raggedright}\gnumbox[l]{Licen�a software para guich�}
	&\gnumericPB{\raggedleft}\gnumbox[r]{600}
	&\gnumericPB{\raggedleft}\gnumbox[r]{$              210 $}
	&\gnumericPB{\raggedleft}\gnumbox[r]{126000}
\\
	 \gnumericPB{\centering}\gnumbox{2015}
	&\gnumericPB{\raggedright}\gnumbox[l]{Licen�a software para �nibus}
	&\gnumericPB{\raggedleft}\gnumbox[r]{4800}
	&\gnumericPB{\raggedleft}\gnumbox[r]{$              66 $}
	&\gnumericPB{\raggedleft}\gnumbox[r]{316800}
\\
 \gnumericPB{\centering}\gnumbox{\textbf{2015}}
	&\gnumericPB{\raggedright}\gnumbox[l]{\textbf{Sub total - 2015}}
	&\gnumericPB{\raggedleft}\gnumbox[r]{}
	&\gnumericPB{\raggedleft}\gnumbox[r]{}
	&\gnumericPB{\raggedleft}\gnumbox[r]{\textbf{442800}}
\\
	 \gnumericPB{\centering}\gnumbox{2016}
	&\gnumericPB{\raggedright}\gnumbox[l]{Licen�a software para guich�}
	&\gnumericPB{\raggedleft}\gnumbox[r]{600}
	&\gnumericPB{\raggedleft}\gnumbox[r]{$              220 $}
		&\gnumericPB{\raggedleft}\gnumbox[r]{132000}
\\
\hhline{-----}
	 \gnumericPB{\centering}\gnumbox{\textbf{Total}}
	&
	&\gnumericPB{\raggedleft}\gnumbox[r]{\textbf{$           5.400 $}}
	&
	&\gnumericPB{\raggedleft}\gnumbox[r]{\textbf{$          926.100 $}}
\\
\hhline{-----}
\end{longtable}

\ifthenelse{\isundefined{\languageshorthands}}{}{\languageshorthands{\languagename}}
\gnumericTableEnd
