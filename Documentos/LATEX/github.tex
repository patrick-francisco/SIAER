%---------- Inicio do Texto ----------
% recomenda-se a escrita de cada capitulo em um arquivo texto separado (exemplo: intro.tex, fund.tex, exper.tex, concl.tex, etc.) e a posterior inclusao dos mesmos no mestre do documento utilizando o comando \input{}, da seguinte forma:
%\input{intro.tex}
%\input{fund.tex}
%\input{exper.tex}
%\input{concl.tex}

\def\inputGnumericTable{}  

\chapter{Desenvolvimento}\label{chap:desenvolvimento}
Este projeto foi baseado no resultado do trabalho de BASSAN JUNIOR, Luiz Alberto e FITZ LUCCETTI, Marcus Vin�cius, o qual foi apresentado em junho de 2010 frente a uma banca de professores da UTFPR como projeto final do curso de engenharia industrial  el�trica com �nfase em eletr�nica e telecomunica��es.

O projeto pode ser dividido em tr�s softwares distintos. O primeiro, o qual fica no guich�, � composto por um banco de dados e um software em C\#. O segundo � a API que fica no �nibus e mostra os resultados para o motorista, tamb�m programado em C\#. O terceiro � o firmware, programado em C.

Com as atualiza��es realizadas, o software do �nibus foi modificado, para melhorar sua interface, e o \textit{firmware} foi praticamente refeito a partir do zero. Todas essas altera��es est�o documentadas neste relat�rio. No entanto, no banco de dados e no software do guich� n�o houveram atualiza��es, sendo praticamente o mesmo usado na �ltima apresenta��o. Para mais informa��es acerca de tais partes do projeto, recomenda-se a leitura do �tem 3.5 do relat�rio apresentado por BASSAN JUNIOR, Luiz Alberto e FITZ LUCCETTI, Marcus Vin�cius para a primeira defesa do projeto.

\section{\textit{GITHUB}}

GitHub � um Servi�o de Web Hosting Compartilhado para projetos que usam o controle de versionamento Git. � escrito em Ruby on Rails pelos desenvolvedores da Logical Awesome (Chris Wanstrath, PJ Hyett e Tom Preston - Wernder). O GitHub possui planos comerciais e gratuitos para projetos de c�digo aberto.

Este site possui funcionalidades de uma rede social como feeds, followers, wiki e um gr�fico que mostra como os desenvolvedores trabalham as vers�es de seus reposit�rios.

Todos os arquivos est�o no GITHUB.